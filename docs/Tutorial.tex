% Created 2019-10-23 Mi 16:32
% Intended LaTeX compiler: pdflatex
\documentclass[11pt]{article}
\usepackage[utf8]{inputenc}
\usepackage[T1]{fontenc}
\usepackage{graphicx}
\usepackage{grffile}
\usepackage{longtable}
\usepackage{wrapfig}
\usepackage{rotating}
\usepackage[normalem]{ulem}
\usepackage{amsmath}
\usepackage{textcomp}
\usepackage{amssymb}
\usepackage{capt-of}
\usepackage{hyperref}
\usepackage[margin=1.5cm]{geometry}
\author{Christopher Ehlert}
\date{\today}
\title{}
\hypersetup{
 pdfauthor={Christopher Ehlert},
 pdftitle={},
 pdfkeywords={},
 pdfsubject={},
 pdfcreator={Emacs 26.3 (Org mode 9.2.6)}, 
 pdflang={English}}
\begin{document}

\tableofcontents


\section{Introduction}
\label{sec:org4c9492c}
This tutorial serves as a practical introduction to PSIXAS and its ability to simulate 
X-ray absorption processes like X-ray photoelectron spectroscopy (XPS), 
near edge X-ray absorption fine structure (NEXAFS) and the pump-probe NEXAFS spectroscopies.

The theoretical foundation used by PSIXAS are the 
transition-potential and \(\Delta\)-Kohn-Sham method. Other, wavefunction based methods 
are planned for future implementations.


\section{Installation}
\label{sec:orge19c54b}
The most convenient way to get PSIXAS to run is to use PSI4 via 
conda or miniconda. Let's assume you created a new environment:

\begin{verbatim}
conda create -n p4env psi4 psi4-dev -c psi4
conda activate p4env
\end{verbatim}
which will install the most recent stable PSI4.

Then you can checkout the PSIXAS repository into the directory where all your PSI4
plugins are located:
\begin{verbatim}
git clone https://github.com/Masterluke87/psixas
cd psixas

$(psi4 --plugin-compile)
make
\end{verbatim}
If you do not want to build the master branch version, you can checkout
 another branch:
\begin{verbatim}
git checkout PSIXAS-1.0
\end{verbatim}

After the plugin one needs to export the path that contains the plugin:
\begin{verbatim}
export PYTHONPATH=/path/to/psi4Plugins
\end{verbatim}
The installation is complete, and PSIXAS is ready to run!

\section{Example 1: XPS calculation}
\label{sec:orgaa672c1}
\subsection{3,3,3-Trifluoropropanol}
\label{sec:orge2d5d52}
For 3,3,3-Trifluoropropanol one would expect three distinct C1s signals in the XPS spectrum, each signal can be assigned 
to one carbon atom in the molecule.

The central quantity one has to calculate when simulating XPS spectra are the so-called core-electron binding energies (CEBEs).
These can be obtain via a \(\Delta\)-Kohn-Sham calculation. The CEBE for an excitation center \(i\) is given 
as CEBE = E\textsuperscript{i}\textsubscript{kat} - E\textsubscript{neu}, where E\textsubscript{neu} and E\textsubscript{kat} are the energies of the neutral 
and core-ionized molecule, respectively.

In this example, we are calculating the core electron binding energies with B3LYP/def2-TZVP. In order to get the energy of the 
neutral molecule, we can use the folling input:
\begin{verbatim}
import psixas

memory 16GB

molecule{
 C           -0.348376548098    -0.548474458915    -0.927836216945
 C            0.598339605925     0.344412190721    -0.154926820251
 C           -1.260377523962    -1.428990131337    -0.073815750620
 H           -0.931459004013     0.095187059173    -1.591052058736
 H            0.270520314607    -1.188949646350    -1.560246367799
 F           -0.030731307342     1.342027128065     0.491810630465
 F            1.492395966346     0.923480284909    -0.991220902737
 F            1.309169804186    -0.344637834439     0.762302565414
 H           -1.792305519830    -2.113716429507    -0.745127437896
 O           -2.163746992897    -0.714605675932     0.750272586920
 H           -0.660551886505    -2.035740348426     0.604285075819
 H           -2.747844602396    -0.181104236356     0.200201028032
symmetry c1
}

set {
  basis def2-TZVP
}

set psixas {
  prefix TRIFLUOR
  MODE GS
 }

set scf {
 reference uks
 scf_type MEM_DF
}

energy('psixas',functional='B3LYP')
\end{verbatim}

\section{Example 2: NEXAFS calculation}
\label{sec:org0c3c98f}


\section{Example 3: PP-NEXAFS calculation}
\label{sec:orgfdce476}
\end{document}